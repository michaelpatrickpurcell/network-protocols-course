\documentclass[aspectratio=169]{beamer}

\usetheme{sii}

\title{SSL/TLS: Encryption}
\author{Michael Purcell}
\date{01 June 2021}
\begin{document}

\begin{frame}[Triangle=siiorange]
\titlepage
\end{frame}

\section{Cipher Suites}
\begin{frame}[Triangle=siiorange]
	\tocpage
\end{frame}

\begin{frame}[triangle=siiblue]
	\frametitle{Cipher Suites}
	A cipher suite is a set of cryptographic algorithms that can be used to secure a SSL/TLS session. An SSL/TLS cipher suite consists of:
	\begin{itemize}
		\item A key exchange algorithm
		\item An authentication algorithm
		\item A bulk encryption algorithm
		\item A message authentication code (MAC) algorithm
	\end{itemize}
	
	\vfill
	
	The primary purpose of the SSL/TLS Handshake is to establish which cipher suite will be used during the subsequent session.
\end{frame}

\begin{frame}[triangle=siiblue]
	\frametitle{Example Cipher Suites}
	Modern cipher suites that are believed to be secure:
	\begin{itemize}
		\item \texttt{TLS\_ECDHE\_ECDSA\_WITH\_AES\_256\_GCM\_SHA384}
		\item \texttt{TLS\_ECDHE\_RSA\_WITH\_AES\_128\_CBC\_SHA256}
		\item \texttt{TLS\_DHE\_RSA\_WITH\_CHACHA20\_POLY1305\_SHA256}
	\end{itemize}
	
	\vfill
	
	Legacy cipher suites that may be insecure:
	\begin{itemize}		
		\item \texttt{TLS\_RSA\_WITH\_RC4\_128\_MD5}
		\begin{itemize}
			\item Both \texttt{RC4} and \texttt{MD5} are known to be insecure and should not be used in modern applications.
		\end{itemize}
		\item \texttt{TLS\_DH\_DSS\_WITH\_3DES\_EDE\_CBC\_SHA}
		\begin{itemize}
			\item The small block size of \texttt{3DES} (64 bits) makes it unsuitable to handle high-volume traffic as as demonstrated by the "Sweet32" attack.
		\end{itemize}
	\end{itemize}
\end{frame}

\section{The Pseudorandom Function}
\begin{frame}[Triangle=siiorange]
	\tocpage
\end{frame}

\begin{frame}[triangle=siiblue]
	\frametitle{HMAC}
	Given a hash function \texttt{hash}, we define
	\begin{equation} \nonumber
		\texttt{HMAC\_hash}(\texttt{k}, \texttt{m}) = \texttt{hash}\left((\texttt{k'} \oplus \texttt{opad}) \ \Vert \  \texttt{hash}\left((\texttt{k'} \oplus \texttt{ipad}) \ \Vert \ \texttt{m} \right)\right)
	\end{equation}
	where $\texttt{opad} = \texttt{0x5c5c\ldots 5c}$, $\texttt{ipad} = \texttt{0x3636\ldots 36}$, and \texttt{k'} is a block-sized key derived from \texttt{k}.  That is
	\begin{align*}
		\texttt{k'} &=
		\begin{cases}
			\texttt{hash}(\texttt{k}) & \text{if \texttt{k} is larger than the block size}; \\
			\texttt{k} & \text{otherwise}.
		\end{cases} \\
	\end{align*}
\end{frame}


\begin{frame}[triangle=siiblue]
	\frametitle{The Pseudorandom Function}
	We have
	\begin{equation} \nonumber
		\texttt{PRF}(\texttt{secret}, \texttt{label}, \texttt{seed}) = \texttt{P\_hash}(\texttt{secret}, \texttt{label} \ \Vert \  \texttt{seed})
	\end{equation}
	where
	\begin{equation} \nonumber
	\begin{split}
		\texttt{P\_hash}(\texttt{secret}, \texttt{seed}) = \ &\texttt{HMAC\_hash}(\texttt{secret}, \texttt{A}(1) \ \Vert \  \texttt{seed})\\
		&\ \Vert \ \texttt{HMAC\_hash}(\texttt{secret}, \texttt{A}(2) \ \Vert \  \texttt{seed}) \\
		&\ \Vert \ \texttt{HMAC\_hash}(\texttt{secret}, \texttt{A}(3) \ \Vert \  \texttt{seed}) \\
		&\qquad \vdots
	\end{split}
	\end{equation}
\end{frame}

\section{Key Calculation}
\begin{frame}[Triangle=siiorange]
	\tocpage
\end{frame}

\begin{frame}[triangle=siiblue]
	\frametitle{Compute the Master Secret}
	Let $\texttt{cs\_random}$ be the concatenation of the random bytes supplied by the client and server during the TLS Handshake. That is
	\begin{equation} \nonumber
	\texttt{cs\_random} = \texttt{client\_random} \ \Vert \ \texttt{server\_random}.
	\end{equation}
	Then we compute the value of \texttt{master\_secret} via
	\begin{align*}
		\texttt{master\_secret} = \texttt{PRF}(&\texttt{pre\_master\_secret},\\
		&\texttt{"master secret"},\\
		&\texttt{cs\_random})[0\ldots 47].
	\end{align*}
	
	\vfill
	
	Notice that \texttt{master\_secret} is always 48 bytes long.
\end{frame}

\begin{frame}[triangle=siiblue]
	\frametitle{Key Calculation}
	Let $\texttt{sc\_random}$ be the concatenation of the random bytes supplied by the server and client during the TLS Handshake. Notice that the order is reversed form that used in the computation of \texttt{master\_secret}. That is
	\begin{equation} \nonumber
	\texttt{sc\_random} = \texttt{server\_random} \ \Vert \ \texttt{client\_random}.
	\end{equation}
	Then we compute the value of \texttt{key\_block} via
	\begin{equation} \nonumber
	\begin{split}
		\texttt{key\_block} = \texttt{PRF}(&\texttt{master\_secret}, \\
		&\texttt{"key expansion"},\\
		&\texttt{sc\_random})[0\ldots \texttt{n-1}]
	\end{split}
	\end{equation}
	where \texttt{n} is the required number of bytes of key material.
\end{frame}

\section{Record Payload Protection}
\begin{frame}[Triangle=siiorange]
	\tocpage
\end{frame}

\subsection{Legacy Cipher Suites}
\begin{frame}[triangle=siiblue]
	\frametitle{MAC-then-encrypt}
	All of the cipher suites defined in the original description of TLSv1.2 use \texttt{HMAC\_hash} as their MAC algorithm.

	\vfill
	
	The MAC is computed on the unencrypted record data.
	
	\vfill
	
	The bulk encryption algorithm is then used to encrypt both the record data and MAC.
	
	\vfill
	
	This is commonly called the MAC-then-Encrypt (MtE) construction.
\end{frame}

\begin{frame}[triangle=siiblue]
	\frametitle{Computing The MAC}
	The MAC is computed on a payload that consists of a sequence number, parameters describing any TLS compression that has been applied, and a TLS record fragment. That is
	\begin{equation}\nonumber
	\begin{split}
		\texttt{payload} = \texttt{seq\_num} &\ \Vert \ \texttt{compressed\_type} \\
		&\ \Vert \ \texttt{compressed\_version} \\
		&\ \Vert \ \texttt{compressed\_length} \\
		&\ \Vert \ \texttt{record\_fragment}.
	\end{split}
	\end{equation}
	
	\vfill
	
	For cipher suites that use the \texttt{HMAC\_hash} as their MAC algorithm, the MAC is given by:
	\begin{equation}\nonumber
		\texttt{MAC} = \texttt{HMAC\_hash}\left(\texttt{MAC\_write\_key}, \texttt{payload}\right).
	\end{equation}
\end{frame}

\begin{frame}[triangle=siiblue]
	\frametitle{Stream Ciphers}
	A \emph{stream cipher} generates a key stream which is XORed with the plaintext to produce the ciphertext.
	
	\vfill
	
	Prior to 2016, RC4 was the only stream cipher used in SSL/TLS cipher suites.
	
	\vfill
	
	In 2016, cipher suites which use ChaCha20 were introduced by RFC 7905.
	
	\vfill
	
	Use of RC4 was prohibited in 2015 by RFC 7465
\end{frame}

\begin{frame}[triangle=siiblue]
	\frametitle{CBC Block Ciphers}
	A block cipher applies a key-dependent permutation to a \emph{block} of plaintext to produce a block of ciphertext.
	
	\vfill
	
	AES is used in many SSL/TLS cipher suites.
	
	\vfill
	
	A \emph{block cipher mode of operation} is an algorithm that allows a block cipher to be used to encrypt a plaintext that consists of more than one block. 
	
	\vfill 
	
	The original specification for TLSv1.2 included algorithms for using block ciphers in CBC (cipher block chaining) mode to encrypt TLS records.
\end{frame}

%\begin{frame}
%	\frametitle{CBC Details}
%	Encryption:
%	
%	\vfill
%	
%	Decryption:
%\end{frame}

\begin{frame}[triangle=siiblue]
	\frametitle{Padding}
	Block ciphers can only be applied to plaintexts whose length is an integer multiple of the block size.
	
	\vfill
	
	If a TLS record is not an appropriate length then \emph{padding} must be added.

	\vfill
	
	The padding may be any length up to 255 bytes.
	
	\vfill
	
	Each byte of padding is filled with the padding length value.
\end{frame}

\begin{frame}[triangle=siiblue]
	\frametitle{Initialization Vector}
	Using a block cipher in CBC mode requires an initialization vector (IV).
	
	\vfill
	
	The IV is not secret and is sent in-the-clear immediately prior to the encrypted contents of its associated TLS record.
	
	\vfill
	
	Despite being sent in-the-clear, the IV for each record must be unpredictable prior to transmission.
	
	\vfill
	
	Early versions of TLS did not use explicit IVs. Instead, they used the last ciphertext block of the previous record. This allows an attacker compromise the security of the connection as demonstrated by  the so-called "BEAST" attack.

\end{frame}
\subsection{AEAD Cipher Suites}
\begin{frame}[triangle=siiblue]
	\frametitle{Authenticated Encryption}
	Authenticated encryption (AE) combines the functionalities of bulk encryption and message authentication into a single algorithm.
	
	\vfill
	
	One simple way to create an authenticated encryption algorithm is via the encrypt-then-MAC (EtM) construction.
	
	\vfill
	
	With EtM, the plaintext is first encrypted and then a MAC is computed (using a different key) for the resulting ciphertext. 
	
\end{frame}

\begin{frame}[triangle=siiblue]
	\frametitle{Additional Data}
	Authenticated encryption with associated data (AEAD) is an extension of authenticated encryption that allows a sender to include some data that must be authenticated but does not need to be (or cannot be) encrypted.
	
	\vfill
	
	One common example of associated data is packet routing information.  If the destination address and/or ports are encrypted, then the packet cannot be delivered.  It can be useful, however, to authenticated the routing information to prevent an attacker from manipulating network traffic flows.
\end{frame}

\begin{frame}[triangle=siiblue]
	\frametitle{AEAD Algorithms}
	AEAD algorithms used in SSL/TLS cipher suites include:
	\begin{itemize}
		\item AES-GCM (AES - Galois/Counter Mode)
		\item AES-CCM (AES - Counter with CBC-MAC)
		\item ChaCha20-Poly1305
	\end{itemize}

	\vfill
	
	The plaintext for a MAC computing these AEAD algorithm consists of a TLS record fragment.
	
	\vfill
	
	The associated data consists of a sequence number and information about any TLS compression that has been applied.
\end{frame}

\section{References}
\begin{frame}[Triangle=siiorange]
	\tocpage
\end{frame}

\begin{frame}[triangle=siiblue]
	\frametitle{RFCs}
	\begin{itemize}
		\item RFC 5246 - ``The Transport Layer Security (TLS) Protocol Version 1.2''
		
		\vfill
				
		\item RFC 2104 - ``HMAC: Keyed-Hashing for Message Authentication''

		\vfill
				
		\item RFC 7465 - ``Prohibiting RC4 Cipher Suites''

		\vfill
				
		\item RFC 7905 - ``ChaCha20-Poly1305 Cipher Suites for Transport Layer Security (TLS)''

		\vfill
				
		\item RFC 5116 - ``An Interface and Algorithms for Authenticated Encryption''
	\end{itemize}
\end{frame}

\begin{frame}[triangle=siiblue]
	\frametitle{Other Resources}
	\begin{itemize}
		\item The paper ``On the Practical (In-)Security of 64-bit Block Ciphers: Collision Attacks on HTTP over TLS and OpenVPN'' by Karthikeyan Bhargavan and Ga\"etan Leurent
		
		\vfill
				
		\item The blog ``Command Line Fanatic'' by Jeremy Davies
		\begin{itemize}
			\item ``An Illustrated Guide to the BEAST Attack''
		\end{itemize}
		
		\vfill
				
		\item The blog ``hashedout'' by The SSL Store
		\begin{itemize}
			\item ``Cipher Suites: Ciphers, Algorithms and Negotiating Security Settings'' by Patrick Nohe
		\end{itemize}
		
		\vfill
		
		\item The blog ``Moxie Marlinspike'' by Moxie Marlinspike
		\begin{itemize}
			\item ``The Cryptographic Doom Principle''
		\end{itemize}
	\end{itemize}
	
\end{frame}
\end{document}